% Overleaf-ready LaTeX Report for the Coconut Harvester Dashboard
% Notes:
% - Minimal dependencies (no shell-escape); uses listings for code.
% - To compile with bibliography, ensure references.bib is uploaded; path here assumes repo layout.
% - If uploading only this file to Overleaf, also upload ../references.bib, or change the path to a local copy.

\documentclass[12pt,a4paper]{report}
\usepackage[margin=1in]{geometry}
\usepackage{graphicx}
\usepackage{amsmath,amssymb}
\usepackage{booktabs}
\usepackage{xcolor}
\usepackage{hyperref}
\usepackage{listings}
\usepackage{caption}
\usepackage{subcaption}
\usepackage{tikz}
\usepackage{enumitem}
\usepackage{siunitx}

\hypersetup{
  colorlinks=true,
  linkcolor=blue,
  citecolor=blue,
  urlcolor=blue
}

\lstdefinestyle{code}{
  language=Python,
  basicstyle=\ttfamily\small,
  keywordstyle=\color{blue!60!black},
  commentstyle=\color{green!40!black},
  stringstyle=\color{red!60!black},
  showstringspaces=false,
  numberstyle=\tiny, numbers=left, numbersep=6pt,
  frame=single, framerule=0.3pt, rulecolor=\color{black!30},
  breaklines=true,
}

\title{Autonomous Coconut Harvester Dashboard\\\large PDE, Transform, and Optimization in a Robotics Context}
\author{Siju K. S.}
\date{October 17, 2025}

\begin{document}
\maketitle
\pagenumbering{roman}

\begin{abstract}
This project presents an interactive, educational simulation of an autonomous coconut-harvesting rover that integrates ideas from Partial Differential Equations (PDEs), Laplace transforms, and Optimization.
The system renders a realistic orchard map, a six-wheel rover with an articulated arm, and real-time telemetry. Planning respects field time, energy, and thermal constraints. The work is designed to reinforce the five course modules through a cohesive, hands-on application.
\end{abstract}

\tableofcontents
\clearpage
\pagenumbering{arabic}

\chapter{Introduction}
\paragraph{Motivation.} Many engineering topics are taught in isolation. This work unifies (i) numerical PDEs, (ii) Laplace-domain modeling, and (iii) optimization, inside a robotics-flavored application: a coconut-harvesting rover.

\paragraph{Contributions.}
\begin{itemize}[noitemsep]
  \item A realistic, Matplotlib-based dashboard with a dominant field view and framed telemetry panels.
  \item A simple mission planner that enforces time and energy constraints while considering obstacles.
  \item A thermal model of the cutting tool using a 1D heat equation (FTCS), yielding safe duty cycles.
  \item A symbolic joint response (step input) to extract a settling time via Laplace-domain reasoning.
  \item A clean mapping of code to five instructional modules with tests and reproducibility guidance.
\end{itemize}

\chapter{System Overview}
\section{Architecture}
Figure~\ref{fig:arch} summarizes the flow. The planner generates a mission queue (drive/scan/deploy/cut/cool) subject to constraints. The main loop updates environment, kinematics, thermal state, and telemetry. The UI presents panels for world, systems, AI, timeline, and logs.

\begin{figure}[h]
  \centering
  \begin{tikzpicture}[node distance=1.6cm, auto, >=latex']
    \tikzstyle{block}=[rectangle, draw, rounded corners, align=center, minimum width=3.2cm, minimum height=0.9cm]
    \node[block, fill=blue!5] (planner) {Planner\\\small optimizer.py};
    \node[block, fill=green!5, right=2.6cm of planner] (mainloop) {Simulation Loop\\\small main.py};
    \node[block, fill=orange!10, below=1.2cm of mainloop] (thermal) {Thermal (FTCS)\\\small thermal.py};
    \node[block, fill=purple!10, above=1.2cm of mainloop] (dynamics) {Arm Step Response\\\small dynamics.py};
    \node[block, fill=cyan!10, right=2.8cm of mainloop] (ui) {Dashboard UI\\\small Matplotlib};
    \node[block, fill=yellow!10, below=1.2cm of planner] (kin) {Kinematics\\\small kinematics.py};

    \draw[->] (planner) -- node[sloped, above]{mission queue} (mainloop);
    \draw[->] (dynamics) -- node[right]{settling time} (mainloop);
    \draw[->] (thermal) -- node[right]{tool temp profile} (mainloop);
    \draw[->] (kin) -- node[sloped, above]{FK, world pose} (mainloop);
    \draw[->] (mainloop) -- node[sloped, above]{telemetry \& visuals} (ui);
  \end{tikzpicture}
  \caption{High-level architecture and data flow.}
  \label{fig:arch}
\end{figure}

\section{Key Files}
\begin{itemize}[noitemsep]
  \item \texttt{main.py}: Orchestrates the simulation, rendering, and telemetry.
  \item \texttt{optimizer.py}: Builds a feasible path under time/energy constraints with obstacle checks.
  \item \texttt{dynamics.py}: Solves a second-order joint ODE and extracts 2\% settling time.
  \item \texttt{kinematics.py}: Planar 2-link forward kinematics and world coordinate projection.
  \item \texttt{thermal.py}: 1D heat equation (FTCS) for drilling/cooling and duty-cycle discovery.
\end{itemize}

\chapter{Methods}
\section{Numerical PDE: 1D Heat Equation (FTCS)}
We model the cutting tool’s temperature along its length using the heat equation
\begin{equation}
  \frac{\partial u}{\partial t} = \alpha\,\frac{\partial^2 u}{\partial x^2} + Q,\qquad x \in [0,L].
\end{equation}
Using FTCS: \(u^{n+1}_i = u^n_i + r\,(u^n_{i+1}-2u^n_i+u^n_{i-1}) + Q\,\Delta t\), with \(r=\alpha\,\Delta t/\Delta x^2 \le 0.5\) for stability. The model heats during cutting (\(Q>0\)) and cools with \(Q=0\). The \texttt{find\_duty\_cycle} routine determines safe drill/cool durations under a temperature cap.

\section{Laplace Domain: Joint Step Response}
A simplified joint obeys \(I\,\theta'' + c\,\theta' + k\,\theta = k\,u(t)\). Solving in the Laplace domain and applying inverse transforms yields \(\theta(t)\) for a unit step. We detect the 2\% settling time numerically and feed it to the planner to time the arm deployment.

\section{Optimization: Feasible Harvest Plan}
The planner evaluates candidates from the current rover position, estimating time and energy for (drive, scan, deploy, cut, cool). Lines-of-sight intersecting circular obstacles are marked blocked. A greedy selection (nearest feasible) continues while respecting global time and energy budgets. The output is a mission path, totals, and a list of blocked/skipped trees.

\chapter{Implementation}
\section{World Rendering and Rover}
The field uses a procedural orchard texture, canopy discs, and coconut scatters. The rover is a polygon chassis with six wheel footprints; the articulated arm overlay is drawn from forward kinematics of a 2-link chain at the rover base.

\section{Telemetry and Panels}
The UI displays: (i) a large world map, (ii) tree tracker, (iii) AI “Mission Brain,” (iv) mission utilization, arm response, tool temperature, (v) systems telemetry, obstacles, (vi) real-time telemetry blocks, mission timeline, summary, and planner console. Telemetry is logged per frame and can be saved to CSV.

\section{Representative Code}
\begin{lstlisting}[style=code, caption={Thermal update in CUT/COOL actions.}]
# main.py (excerpt)
if action == 'CUT_TREE':
    # Heating phase
    tool_temp_profile = thermal.solve_heat_equation(
        elapsed_dt, Q=60, initial_temp_profile=tool_temp_profile)
elif action == 'COOL_TOOL':
    # Cooling phase
    tool_temp_profile = thermal.solve_heat_equation(
        elapsed_dt, Q=0, initial_temp_profile=tool_temp_profile)
\end{lstlisting}

\chapter{Results}
\section{Functional Outcomes}
The simulation runs interactively or headlessly, generating a feasible path with status coloring for trees (pending, scanning, arm\_ready, harvested, empty, blocked, skipped). The AI panel surfaces intent, risk level, and next action. Telemetry captures environment, drivetrain, manipulator, comms/nav, tasking, and AI labels.

\section{Testing}
Unit tests verify: (i) thermal monotonic cooling and heating energy rise, (ii) planner constraint adherence and obstacle blocking, (iii) dashboard headless initialization and CSV logging. In our run, all tests passed (6/6).

\chapter{Telemetry Schema}
Table~\ref{tab:telemetry} lists key fields. Vector-valued entries are serialized as comma-joined strings.

\begin{table}[h]
  \centering
  \small
  \begin{tabular}{ll}
    \toprule
    Field & Description \\
    \midrule
    time & Simulation time (s) \\
    action, tree & Current action and target tree \\
    rover\_x, rover\_y & Rover position (m) \\
    energy & Cumulative energy used (J) \\
    max\_tool\_temp & Peak blade temperature (\si{\celsius}) \\
    battery\_pct, pack\_V/A & Battery state, pack voltage/current \\
    wheel\_currents/torques/temps & Per-wheel metrics \\
    suspension\_loads & Per-suspension loads (kN) \\
    cutter\_rpm & Cutter speed (rpm) \\
    arm\_joint\_angles/temps & 6-DOF angles (rad) and joint temperatures (\si{\celsius}) \\
    end\_effector\_force & Force at tool (N) \\
    nav\_confidence, risk & Navigation confidence and AI risk label \\
    \bottomrule
  \end{tabular}
  \caption{Selected telemetry fields written per frame.}
  \label{tab:telemetry}
\end{table}

\chapter{Module Mapping (01--05)}
\begin{description}[style=unboxed,leftmargin=0cm]
  \item[01: Intro to Python \'\& Scientific Stack] Arrays, plotting, and kinematics warm-ups echo \texttt{kinematics.py}.
  \item[02: PDE Numerical] Upwind advection and wave equation complement the FTCS heat solver \texttt{thermal.py}.
  \item[03: Laplace Basics] Frequency response and time-domain pairs underpin the joint step response \texttt{dynamics.py}.
  \item[04: Laplace Applications] Piecewise/impulse inputs relate to mission phases and switching behavior.
  \item[05: Optimization] Linear programming motifs motivate constrained plan generation in \texttt{optimizer.py}.
\end{description}

\chapter{Discussion and Future Work}
\paragraph{Limitations.} Greedy planning may be suboptimal; terrain and sensing are simplified; the thermal model is 1D and phenomenological.
\paragraph{Extensions.} Replace greedy with MILP/TSP variants; introduce stochastic sensing; add dynamics-based control policies; export richer telemetry (Parquet/JSON) and a replay tool.

\chapter{Reproducibility}
\begin{itemize}[noitemsep]
  \item Python \texttt{3.11+}; install dependencies from \texttt{requirements.txt}.
  \item Run tests with \texttt{pytest}; headless mode writes telemetry CSV for CI checks.
  \item Deterministic runs via \texttt{--seed}.
\end{itemize}

\chapter{Conclusion}
The dashboard links PDEs, Laplace transforms, and optimization into a coherent robotics application. It serves as a capstone for computational engineering methods with practical visuals, telemetry, and tests.

\chapter{References}
\begin{enumerate}[label={[R\arabic*]},leftmargin=*,itemsep=0.25em]
  \item Project repository: \url{https://github.com/sijuswamyresearch/PDE-T-and-OT}
  \item Course book configuration (Quarto): \texttt{\_quarto.yml} and modules \texttt{01--05 .qmd} in the repository.
  \item Core implementation files: \texttt{main.py}, \texttt{optimizer.py}, \texttt{dynamics.py}, \texttt{kinematics.py}, \texttt{thermal.py}.
  \item Matplotlib, NumPy, SymPy documentation for API usage.
\end{enumerate}

\end{document}
